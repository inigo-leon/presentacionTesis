%&latex

\documentclass[size=12pt,display=slidesnotes]{powerdot}

\usepackage[english]{babel}
\usepackage{multicol}
\usepackage{amsmath}
\usepackage{amsthm}
\usepackage{amssymb}
\usepackage{graphicx}
\usepackage{amsfonts}
\usepackage{kbordermatrix}
\usepackage{alltt}
\usepackage{ulem}

\usepackage{listings}
\usepackage{lstpatch}

%Para poner colores en los listados xml
\usepackage[usenames,dvipsnames,svgnames,table]{xcolor}
\definecolor{Maroon}{RGB}{255,100,0}


%Definir las caracteristicas de los listados
\lstdefinelanguage{XML}
{
  basicstyle=\ttfamily\scriptsize,
  morestring=[b]",
  moredelim=[s][\bfseries\color{Maroon}]{<}{\ },
  moredelim=[s][\bfseries\color{Maroon}]{</}{>},
  moredelim=[l][\bfseries\color{Maroon}]{/>},
  moredelim=[l][\bfseries\color{Maroon}]{>},
  morecomment=[s]{<?}{?>},
  morecomment=[s]{<!--}{-->},
  commentstyle=\color{DarkOliveGreen},
  stringstyle=\color{blue},
  identifierstyle=\color{red},
}
\lstset{language=XML,
        columns=flexible,
        breaklines=true
        }

%Para usar m�ltiples columnas
\usepackage{multicol} 
\setlength{\columnsep}{7mm}

%para lineas intermitentes
\usepackage{arydshln}

\begin{document}

%+Title
\title{SECURITY IN PETRI NETS SHARING AND STORAGE: SUBNETS, PRIVACY, INTEGRITY,
AUTHENTICATION AND NON REPUDIATION}


\author{\textbf{I\~nigo Le\'on Samaniego}
\\
\\\textbf{Defensa de Tesis Doctoral}\\
Universidad de La Rioja
\\
\\
\begin{small}Directores: Emilio Jim\'enez Mac\'ias,
Juan Ignacio Latorre Biel\end{small}
\\
}

\maketitle
%-Title



\section{Introducci\'on}
\begin{slide}{Entorno}
\end{slide}






\begin{slide}{Justificaci\'on de la investigaci\'on}
\end{slide}




\begin{slide}{Metodolog\'ia}
\end{slide}




\begin{slide}{Alcance}
\end{slide}




\section{Subredes}
\begin{slide}{Conceptos y Definiciones}
\end{slide}




\begin{slide}{Descomposici\'on en subredes}
\end{slide}



\begin{slide}[toc=Front-ends de entrada/salida]{Definici\'on de front-ends de entrada/salida}
\end{slide}




\begin{slide}{Revisi\'on general de PNML}
\end{slide}




\begin{slide}[toc=Extensi\'on de subredes para PNML]{Extensi�n de PNML para soporte de subredes con front-ends}
\end{slide}






\section{Seguridad en Redes de Petri}
\begin{slide}{Alcance}
\end{slide}



\begin{slide}{XMLEncryption}
\end{slide}




\begin{slide}{XMLSignature}
\end{slide}



\begin{slide}{Seguridad completa}
\end{slide}


\section{Ejemplos}
\begin{slide}{Ejemplos}
\end{slide}










\section{BORRARRRRRRR!!!!!}
\begin{slide}[method=file]{codigo XML}
\begin{lstlisting}
<?xml version="1.0" encoding="UTF-8"?>
<pnml
    xmlns="http://www.pnml.org/version-2009/grammar/pnml">
  <net id="5Philo"
      type="http://www.pnml.org/version-2009/grammar/ptnet">
    <page id="page1">
      <place id="p7">
        <name>
          <text>E2</text>
        </name>
\end{lstlisting}
\begin{multicols}{2}
\begin{lstlisting}[basicstyle=\ttfamily\tiny]
<?xml version="1.0" encoding="UTF-8"?>
<pnml xmlns="http://www.pnml.org/version-2009/grammar/pnml">
  <net id="5Philo" type="http://www.pnml.org/version-2009/grammar/ptnet">
    <page id="page1">
      <place id="p7">
        <name>
          <text>E2</text>
        </name>
\end{lstlisting}

\end{multicols}
\end{slide}




\end{document}


